

%  article.tex (Version 3.3, released 19 January 2008)
%  Article to demonstrate format for SPIE Proceedings
%  Special instructions are included in this file after the
%  symbol %>>>>
%  Numerous commands are commented out, but included to show how
%  to effect various options, e.g., to print page numbers, etc.
%  This LaTeX source file is composed for LaTeX2e.

%  The following commands have been added in the SPIE class 
%  file (spie.cls) and will not be understood in other classes:
%  \supit{}, \authorinfo{}, \skiplinehalf, \keywords{}
%  The bibliography style file is called spiebib.bst, 
%  which replaces the standard style unstr.bst.  

\documentclass[a4paper]{spie}  %>>> use for US letter paper
%%\documentclass[a4paper]{spie}  %>>> use this instead for A4 paper
%%\documentclass[nocompress]{spie}  %>>> to avoid compression of citations
%% \addtolength{\voffset}{9mm}   %>>> moves text field down
%% \renewcommand{\baselinestretch}{1.65}   %>>> 1.65 for double spacing, 1.25 for 1.5 spacing 
%  The following command loads a graphics package to include images 
%  in the document. It may be necessary to specify a DVI driver option,
%  e.g., [dvips], but that may be inappropriate for some LaTeX 
%  installations. 
\usepackage{amsmath}
\usepackage[]{graphicx}
\usepackage{hyperref}
\usepackage{listings}
\hypersetup{
    colorlinks,
    linkcolor={black!50!black},
    citecolor={blue!50!black},
    urlcolor={blue!80!black}
}
\usepackage{pdfpages}
\usepackage{parcolumns}

\usepackage[utf8]{inputenc} 
\usepackage[ngerman]{babel}
\usepackage[autostyle=true,german=quotes]{csquotes}

\usepackage{color}
\definecolor{lightgray}{rgb}{.9,.9,.9}
\definecolor{darkgray}{rgb}{.4,.4,.4}
\definecolor{purple}{rgb}{0.65, 0.12, 0.82}


\usepackage{array}
\newcolumntype{L}[1]{>{\raggedright\let\newline\\\arraybackslash\hspace{0pt}}m{#1}}
\newcolumntype{C}[1]{>{\centering\let\newline\\\arraybackslash\hspace{0pt}}m{#1}}
\newcolumntype{R}[1]{>{\raggedleft\let\newline\\\arraybackslash\hspace{0pt}}m{#1}}

\usepackage{xcolor,colortbl}
\usepackage{color}
\usepackage{float}

\title{Smartphonecontroller Javascript-GameAPI}

%>>>> The author is responsible for formatting the 
%  author list and their institutions.  Use  \skiplinehalf 
%  to separate author list from addresses and between each address.
%  The correspondence between each author and his/her address
%  can be indicated with a superscript in italics, 
%  which is easily obtained with \supit{}.

\author{ Ron Schiwkowksi  (918691), Hannes Grothknopf (915449), Michael Schleiss (923739), Christian Heinrichs (919020), Dennis Hofmann (919285)
\skiplinehalf
University of Applied Sciences, Sokratesplatz 1, 24149 Kiel, Germany
}

%%%%%%%%%%%%%%%%%%%%%%%%%%%%%%%%%%%%%%%%%%%%%%%%%%%%%%%%%%%%% 
%>>>> uncomment following for page numbers
\pagestyle{plain}    
%>>>> uncomment following to start page numbering at 301 
%\setcounter{page}{301} 
 
  \begin{document} 
  \maketitle 
%%%%%%%%%%%%%%%%%%%%%%%%%%%%%%%%%%%%%%%%%%%%%%%%%%%%%%%%%%%%% 
\begin{abstract}
Lorem ipsum dolor sit amet, consetetur sadipscing elitr, sed diam nonumy eirmod tempor invidunt ut labore et dolore magna aliquyam erat, sed diam voluptua. At vero eos et accusam et justo duo dolores et ea rebum. Stet clita kasd gubergren, no sea takimata sanctus est Lorem ipsum dolor sit amet. Lorem ipsum dolor sit amet, consetetur sadipscing elitr, sed diam nonumy eirmod tempor invidunt ut labore et dolore magna aliquyam erat, sed diam voluptua. At vero eos et accusam et justo duo dolores et ea rebum. Stet clita kasd gubergren, no sea takimata sanctus est Lorem ipsum dolor sit amet.
\end{abstract}

%>>>> Include a list of keywords after the abstract 
\keywords{Smartphone, JavaScript, HTML5, Nodejs, API}

%%%%%%%%%%%%%%%%%%%%%%%%%%%%%%%%%%%%%%%%%%%%%%%%%%%%%%%%%%%%%
\section{Einleitung}
Lorem ipsum dolor sit amet, consetetur sadipscing elitr, sed diam nonumy eirmod tempor invidunt ut labore et dolore magna aliquyam.

\subsection{Problemstellung}
Lorem ipsum dolor sit amet, consetetur sadipscing elitr, sed diam nonumy eirmod tempor invidunt ut labore et dolore magna aliquyam erat, sed diam voluptua. At vero eos et accusam et justo duo dolores et ea rebum. Stet clita kasd gubergren, no sea takimata sanctus est Lorem ipsum dolor sit amet. Lorem ipsum dolor sit amet, consetetur sadipscing elitr, sed diam.
\begin{figure}[h!]
	\centering
		\fbox{\includegraphics[width=12cm]{./images/FrontendInit.png}}
		\caption{Nicht ausgelastet Bar}
		\label{fig:FrontendInit}
\end{figure}

Lorem ipsum dolor sit amet, consetetur sadipscing elitr, sed diam nonumy eirmod tempor invidunt ut labore et dolore magna aliquyam erat, sed diam voluptua. At vero eos et accusam et justo duo dolores et ea rebum. Stet clita kasd gubergren, no sea takimata sanctus est Lorem ipsum dolor sit amet. Lorem ipsum dolor sit amet, consetetur sadipscing elitr, sed diam nonumy eirmod tempor invidunt ut labore et dolore magna aliquyam erat, sed diam voluptua. At vero eos et accusam et justo duo dolores et ea rebum. Stet clita kasd gubergren, no sea takimata sanctus est Lorem ipsum dolor sit amet.(siehe Abbildung \ref{fig:FrontendInit}).
\\
Lorem ipsum dolor sit amet, consetetur sadipscing elitr, sed diam nonumy eirmod tempor invidunt ut labore et dolore magna aliquyam erat, sed diam voluptua. At vero eos et accusam et justo duo dolores et ea rebum. Stet clita kasd gubergren, no sea takimata sanctus est Lorem ipsum dolor sit amet. Lorem ipsum dolor sit amet, consetetur sadipscing elitr, sed diam.

\section{Methods}
\subsection{Projektmanagement}
Die Verteilung der Aufgaben ist in Tabelle \ref{table:distribution of tasks} aufgestellt.
Aufgaben sind im Gitlab erfasst und Personen zugewiesen. Ein wöchentliches persönliches Treffen aller Projektmitglieder führt zu einer guten Arbeitsmoral und hohen Effektivität.
\paragraph{GitLab}
Das Projket wurde auf einem eigenen GitLab gehostet und den Projektmitgliedern zur Verfügung gestellt. Die Issues des Systems wurden genutzt um Aufgaben zu Verteilung und zu dokumentieren.\\
Strukuriert wurde das Arbeiten nach Feature Branches, um das Arbeiten am Projekt möglichst flexibel zu halten. Der Master-Branch diente als Release-Branch, in den regelmäßig die Änderungen per Merge Request eingetragen und getestet wurden.
\definecolor{blau}{HTML}{217AA2}
\begin{table}
	\label{table:distribution of tasks}
	\centering
		\caption{Matrix distribution of tasks}
		\begin{tabular}{| L{2.6cm} | C{2cm} | C{2cm} | C{2cm} | C{2cm} | C{2cm} |}
		\hline
		\rowcolor{blau}
		\textcolor{white}{\textbf{Aufgabe}} & \textcolor{white}{\textbf{Schleiss}} & \textcolor{white}{\textbf{Grothknopf}} & \textcolor{white}{\textbf{Schiwkowksi}} & \textcolor{white}{\textbf{Hofmann}} & \textcolor{white}{\textbf{Heinrichs}}\\\hline
		Recherche 		& X & X	& X	& X	& X	\\\hline
		Konzept 		& X & X	& X	& X	& X	\\\hline
		Serverdesign	& 	& X & 	& X	& X \\\hline
		Networking		& 	& X	& 	& X	& X	\\\hline
		GameAPI 		& 	& X	&	&	& X	\\\hline
		DemoGame 		& X	&	&	&	& X	\\\hline
		ClientFrontend 	& X	&	& X	&	&	\\\hline
		ClientBackend 	& X	&	& X	&	&	\\\hline
		Database 	    & 	&	&	& X	& 	\\\hline
		LoginHandling   & 	& 	& 	& X	& 	\\\hline
		Poster          & 	& 	& 	& 	& 	\\\hline
		Documentation   & 	& 	& 	& 	& X \\\hline
		Projektmgmt. 	&	&	&	& 	& X	\\\hline
	\end{tabular} 
\end{table}

\subsection{Technologies}
\subsubsection{NodeJS}
Was ist NodeJS, Warum genutzt
\begin{figure}[h!]
	\centering
		\fbox{\includegraphics[width=12cm]{./images/FrontendInit.png}}
		\caption{Systemarchitektur WiFi Tracking}
		\label{fig:sysArch}
\end{figure}

Die Berechnung der Signalstärke wird nicht durch Kismet vorgenommen, sondern geschieht intern in der jeweiligen Netzwerkkarte. Es wird bei den Raspberry Pi ein LOGILINK WL0084B WiFi-USB-Stick verwendet. Die Dämpfung kann bei verschieden Netzwerkkarten stark abweichen. Deshalb ist es erforderlich, dass im Versuchsaufbau nur gleiche Netzwerkkarten verwendet werden.

\subsubsection{HTML 5}
Was kann HTML5, was nutzen wir für HTML 5 Features
\begin{figure}[h!]
	\centering
		\fbox{\includegraphics[width=10cm]{./images/FrontendInit.png}}
		\caption{Virtuelle Maschinen vs. Docker Container\cite{dockercontainer}}
		\label{fig:dockerVM}
\end{figure}

\subsubsection{Jacascript}
Was kann Javascript, ECMA 6
\begin{figure}[h!]
	\centering
		\fbox{\includegraphics[width=10cm]{./images/FrontendInit.png}}
		\caption{Virtuelle Maschinen vs. Docker Container\cite{dockercontainer}}
		\label{fig:dockerVM}
\end{figure}
\subsubsection{MongoDB}
Was kann MongoDB, Warum MongoDB
\begin{figure}[h!]
	\centering
		\fbox{\includegraphics[width=10cm]{./images/FrontendInit.png}}
		\caption{Virtuelle Maschinen vs. Docker Container\cite{dockercontainer}}
		\label{fig:dockerVM}
\end{figure}
\subsubsection{Hardware}
Smartphone Controller, Platformunabhängig Node Server

%
% -----------------Implementation-------------------------
%

\section{Implementation}

\subsection{User Management}
Das "UserManagement" kümmert sich um die Verwaltung der Nutzer. Dafür stellt es Funktionen für die Registrierung, Authentifizierung sowie für das setzen und holen von Benutzerdaten zur Verfügung. Die Funktionen im "UserManagement" bekommen eine callback Funktion übergeben, welche bei einer erfolgreichen Ausführung ausgeführt wird.


\subsection{Database}
Das "Database" Modul kümmert sich um die Persistente Speicherung aller anfallenden Daten. Dafür wird eine MongoDB verwendete und die für das hinzufügen, löschen, updaten und für Abfragen benötigten Funktionen zur Verfügung gestellt.


\subsection{API Beschreibung}
Hier wird die API Beschrieben!

\section{Ausblick}

\paragraph{Glättung der Sensor Werte} Ein noch zu lösendes Problem ist die Glättung der von den Sensoren kommenden Werte. Je nach Gerät liefern die Sensoren unterschiedlich Stark schwankenden Daten. So feuert ein auf dem Tisch liegendes Gerät permanent Orientierungsdaten, obwohl es nicht bewegt wird. Diese Schwankungen müssen erkannt und unterbunden werden. Einbauen eines thresholds.

\paragraph{Begrenzung der gesendeten Pakete}




\paragraph{Disable touch zoom auf dem Client Gerät}



\section{Conclusion}
Alles super. Die erste Million ist nah!




%%%%%%%%%%%%%%%%%%%%%%%%%%%%%%%%%%%%%%%%%%%%%%%%%%%%%%%%%%%%%
%%%%% References %%%%%
\bibliographystyle{spiebib}   %>>>> makes bibtex use spiebib.bst
\bibliography{references}   %>>>> bibliography data in report.bib

\newpage
\appendix

\end{document} 
